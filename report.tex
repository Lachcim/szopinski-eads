\documentclass{report}

\title{
	Singly-linked list of two data types \\
	\large An EADS generic programming project report
}
\date{November 2019}
\author{Michał Szopiński}

\begin{document}

\maketitle

\section{Overview}

The goal of the project was to write a singly-linked list of nodes capable of
storing data entries consisting of two different data types, labeled {\it key} and {\it info}.
This was to be achieved using the class template mechanism offered by the C++
programming language. The end goal was to design a container class whose public
members would facilitate the implementation of a function named {\tt split},
which divides the container into two smaller ones, following a set of rules.

The container in question is named {\tt Sequence} and can be used as follows:

\begin{verbatim}
Sequence<const char*, int> mySequence;
mySequence.add("key 1", 123);
mySequence.add("key 2", 456);
mySequence.add("key 3", 789);
mySequence.remove(1);
\end{verbatim}

The above code allocates a new instance of {\tt Sequence} on the stack,
adds three elements to it, and then promptly removes the second element.
This particular instance is capable of holding keys of type {\tt const char*}
and info of type {\tt int}.

There are many other methods and members of the {\tt Sequence} class template
which can be used to manipulate the container.

\subsection{Memory management}
\subsection{Tail keeping mechanism and iterators}
\subsection{Template implementation detail}

\end{document}
